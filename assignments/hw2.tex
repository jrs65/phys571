\documentclass[12pt]{article}
\usepackage{amsmath}
\usepackage{amssymb}
\usepackage[T1]{fontenc} % Use 8-bit encoding that has 256 glyphs
\usepackage{lmodern}
\usepackage{graphicx}
\usepackage{hyperref}

\usepackage{siunitx}
\sisetup{separate-uncertainty,per-mode=symbol,binary-units}
\DeclareSIUnit\Msolar{M_\odot}
\DeclareSIUnit\degr{deg}
\DeclareSIUnit\parsec{pc}
\DeclareSIUnit\dBm{dBm}
\DeclareSIUnit\jansky{Jy}
\DeclareSIUnit\beam{beam}
\DeclareSIUnit\h{h}

\author{Tristan Pinsonneault-Marotte, Richard Shaw}
\title{PHYS 571: Homework 2}
\date{\today}

\newcommand\diff{\mathrm{d}}

\begin{document}

\maketitle

Deadline: Wednesday 2nd March February, 11:59 PM Pacific Time

\section{Type IA Supernovae analysis mini-project}

In this question we use the standard shorthand for the Hubble parameter $h$ defined as $H_0 = 100\,h\:\si{\kilo\metre\second^{-1}\per\mega\parsec}$.

\paragraph{a)}

Write a routine to numerically calculate the comoving distance to specified redshifts given the cosmological parameters $(h, \Omega_m, \Omega_\Lambda)$, and plot the comoving, angular diameter and luminosity distances from redshifts $z = 0$--$3$ for $h = 0.7$, $\Omega_m = 0.3$ and $\Omega_k = 0$. Comment on why the angular diameter distance is not monotonic.

\emph{Tips: to efficiently calculate the comoving distance for a large number of redshifts in one step, you might consider using \texttt{scipy.integrate.solve\_ivp}.
Also, for testing your code, here are a few precalculated values
\begin{itemize}
    \item $D_L(z = 1, h = 0.7, \Omega_m = 0.3, \Omega_\Lambda = 0.2) = \SI{6050.2}{\mega\parsec}$.
    \item $D_L(z = 1.5, h = 0.75, \Omega_m = 0.4, \Omega_\Lambda = 0.9) = \SI{9921.1}{\mega\parsec}$.
\end{itemize}
}


\paragraph{b)}
Describe how type IA supernovae are detected, and how they are standardised to infer an absolute brightness. [3--4 paragraphs maximum]

\paragraph{c)}
Download the Union2.1 supernova data from here \url{https://supernova.lbl.gov/Union/}. You'll want the \emph{Compilation Magnitude vs. Redshift Table}, which gives the measured redshift and \emph{distance modulus} for 580 supernova, and the \emph{Covariance Matrix with Systematics}. This supernova data has already been standardised and so can be used directly.

Using your code from part a) construct a $\chi^2$ function including the systematic errors for the Union2 supernova data in terms of $(\Omega_m, \Omega_\Lambda, h)$, and find the best fit parameters. \emph{[Tip: for checking your work I find the minimum $\chi^2$ to be $\sim 545$.]}

\paragraph{d)}
Plot the probability distribution $\mathcal{P}(\Omega_m, \Omega_\Lambda)$. You may assume that the noise probability is Gaussian, i.e.
\begin{equation}
\mathcal{P}(d \mid \Omega_m, \Omega_\Lambda, h) \propto \exp{\left(-\frac{1}{2}\chi^2(\Omega_m, \Omega_\Lambda, h)\right)}
\end{equation}
and that there are flat priors on all parameters. Comment on the degeneracy between these parameters and how it could be broken.

\emph{Tip: there are several ways you could do this, either an MCMC sampling scheme (see packages like \texttt{emcee} and \texttt{corner} for plotting), or evaluating the probability directly on a grid. If you struggle with compute time instead of marginalising over $h$ you may just fix $h$ to be the best-fit value from part c)}

\paragraph{e)}
Show the $\Omega_\Lambda = 0$ is very is unlikely by a statistical method of your choosing.

\emph{Tip: suggestions for methods: likelihood ratio test; marginalise to get $\mathcal{P}(\Omega_\Lambda)$; ...}

\end{document}