\documentclass[12pt]{article}
\usepackage{amsmath}
\usepackage{amssymb}
\usepackage[T1]{fontenc} % Use 8-bit encoding that has 256 glyphs
\usepackage{lmodern}
\usepackage{graphicx}
\usepackage{hyperref}

\usepackage{siunitx}
\sisetup{separate-uncertainty,per-mode=symbol,binary-units}
\DeclareSIUnit\Msolar{M_\odot}
\DeclareSIUnit\degr{deg}
\DeclareSIUnit\parsec{pc}
\DeclareSIUnit\dBm{dBm}
\DeclareSIUnit\jansky{Jy}
\DeclareSIUnit\beam{beam}
\DeclareSIUnit\h{h}

\author{Tristan Pinsonneault-Marotte, Richard Shaw}
\title{PHYS 571: Homework 2}
\date{\today}

\newcommand\diff{\mathrm{d}}

\begin{document}

\maketitle

Deadline: Wednesday 2nd March, 11:59 PM Pacific Time

In this assignment you can set $c = \hbar = k_B = 1$.

\section{Flatness Problem}

(Baumann PS1.3) Consider an FRW model dominated by a perfect fluid with pressure
$P = w\rho$, for $w = const$. Let the time-dependent density parameter be
\begin{equation}
    \Omega(t) \equiv \frac{\rho(t)}{\rho_{crit}(t)} \text{ ,}
\end{equation}
where $\rho_{crit}(t) \equiv 3 H^2 / 8 \pi G$. Show that
\begin{equation}
    \frac{d \Omega}{d \ln a} = \left(1 +  3w\right) \Omega
    \left(\Omega - 1\right) \text{ .}
\end{equation}
Discuss the evolution of $\Omega(a)$ for different values of $w$. What happens
to small deviations from flatness $\Omega = 1 + \Omega_k$?


\section{Einstein's Biggest Blunder}

(Baumann PS1.4)

\paragraph{(a)} Consider a universe filled with a perfect fluid with $\rho > 0$
and $P \geq 0$. Show there is no static isotropic homogeneous solution to
Einstein's equations.

\paragraph{(b)} Now, consider a universe filled with pressureless matter ($P_m =
0$) and allow for a cosmological constant $\Lambda$ in the Einstein equation
$G_{\mu\nu} - \Lambda g_{\mu\nu} = 8\pi G T_{\mu\nu}$. Show that it is possible to obtain a static solution if
\begin{equation}
    \Lambda = 4 \pi G \rho_{m,0} \text{ .}
\end{equation}
However, show that this solution is unstable to small perturbations $\delta
\rho_m \ll \rho_m$.

\paragraph{(c)} Is the static universe flat or curved? If it is curved, what is
it's radius of curvature?


\section{The Expanding Universe}

The relative abundances of the different components of energy density determine
the dynamics of the universe, so tracking their evolution is important to
understand the different stages of expansion. Today, the universe is flat and
made up almost entirely of the cosmological constant $\Omega_{\Lambda, 0} =
0.7$, and matter $\Omega_{m,0} = 0.3$. Radiation is a negligible fraction of the
total energy density now, but early on the universe was radiation-dominated.

\paragraph{(a)} Calculate the redshift at which the density of matter and
radiation were equal, given the measured temperature of the CMB today $T_0 =
2.73 K$. Note that at early times, the energy density in neutrinos is related to
the energy density of photons as $\rho_\nu = 3 (7/8) (4/11)^{4/3} \rho_\gamma$
(see Dodelson 2.4.4 for an explanation).

\paragraph{(b)} Calculate the redshift at which the density of the cosmological
constant overtakes the density of matter.

\paragraph{(c)} Calculate the redshift at which the expansion begins to
accelerate.

\section{Type IA Supernovae analysis mini-project}

In this question we use the standard shorthand for the Hubble parameter $h$ defined as $H_0 = 100\,h\:\si{\kilo\metre\second^{-1}\per\mega\parsec}$.

\paragraph{a)}

Write a routine to numerically calculate the comoving distance to specified redshifts given the cosmological parameters $(h, \Omega_m, \Omega_\Lambda)$, and plot the comoving, angular diameter and luminosity distances from redshifts $z = 0$--$3$ for $h = 0.7$, $\Omega_m = 0.3$ and $\Omega_k = 0$. Comment on why the angular diameter distance is not monotonic.

\emph{Tips: to efficiently calculate the comoving distance for a large number of redshifts in one step, you might consider using \texttt{scipy.integrate.solve\_ivp}.
Also, for testing your code, here are a few precalculated values
\begin{itemize}
    \item $D_L(z = 1, h = 0.7, \Omega_m = 0.3, \Omega_\Lambda = 0.2) = \SI{6050.2}{\mega\parsec}$.
    \item $D_L(z = 1.5, h = 0.75, \Omega_m = 0.4, \Omega_\Lambda = 0.9) = \SI{9921.1}{\mega\parsec}$.
\end{itemize}
}


\paragraph{b)}
Describe how type IA supernovae are detected, and how they are standardised to infer an absolute brightness. [3--4 paragraphs maximum]

\paragraph{c)}
Download the Union2.1 supernova data from here \url{https://supernova.lbl.gov/Union/}. You'll want the \emph{Compilation Magnitude vs. Redshift Table}, which gives the measured redshift and \emph{distance modulus} for 580 supernova, and the \emph{Covariance Matrix with Systematics}. This supernova data has already been standardised and so can be used directly.

Using your code from part a) construct a $\chi^2$ function including the systematic errors for the Union2 supernova data in terms of $(\Omega_m, \Omega_\Lambda, h)$, and find the best fit parameters. \emph{[Tip: for checking your work I find the minimum $\chi^2$ to be $\sim 545$.]}

\paragraph{d)}
Plot the probability distribution $\mathcal{P}(\Omega_m, \Omega_\Lambda)$. You may assume that the noise probability is Gaussian, i.e.
\begin{equation}
\mathcal{P}(d \mid \Omega_m, \Omega_\Lambda, h) \propto \exp{\left(-\frac{1}{2}\chi^2(\Omega_m, \Omega_\Lambda, h)\right)}
\end{equation}
and that there are flat priors on all parameters. Comment on the degeneracy between these parameters and how it could be broken.

\emph{Tip: there are several ways you could do this, either an MCMC sampling scheme (see packages like \texttt{emcee} and \texttt{corner} for plotting), or evaluating the probability directly on a grid. If you struggle with compute time instead of marginalising over $h$ you may just fix $h$ to be the best-fit value from part c)}

\paragraph{e)}
Show the $\Omega_\Lambda = 0$ is very is unlikely by a statistical method of your choosing.

\emph{Tip: suggestions for methods: likelihood ratio test; marginalise to get $\mathcal{P}(\Omega_\Lambda)$; ...}

\end{document}