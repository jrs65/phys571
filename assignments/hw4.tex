\documentclass[12pt]{article}
\usepackage{amsmath}
\usepackage{amssymb}
\usepackage[T1]{fontenc} % Use 8-bit encoding that has 256 glyphs
\usepackage{lmodern}
\usepackage{graphicx}
\usepackage{hyperref}
%!TEX root=beamforming.tex

%%% Standard math package imports
\usepackage{amsmath}
\usepackage{bm}
\usepackage{mathtools}



%%% Derivatives
\newcommand\pdiff[1]{\frac{\partial}{\partial #1}}
\newcommand\diff[2]{\frac{d #1}{d #2}}
\newcommand{\curl}{\,\mbox{curl}\,}
\newcommand\del{\nabla}
\newcommand{\grad}[1][]{\ensuremath{\mathbf{\nabla} #1}}



%% Macros for inserting various types of brackets
\newcommand{\la}{\ensuremath{\left\langle}}  % Angle brackets
\newcommand{\ra}{\ensuremath{\right\rangle}}

\newcommand{\lv}{\ensuremath{\left\lvert}}   % Vertical bars
\newcommand{\rv}{\ensuremath{\right\rvert}}

\newcommand{\ls}{\ensuremath{\left[}}        % Square brackets
\newcommand{\rs}{\ensuremath{\right]}}

\newcommand{\lp}{\ensuremath{\left(}}        % Parentheses
\newcommand{\rp}{\ensuremath{\right)}}

%% Macros for equations
\def\beq{\begin{equation}}
\def\eeq{\end{equation}}


%%% Macros for linear algebra notation
\renewcommand{\vec}[1]{{\boldsymbol{\mathbf{#1}}}}
\newcommand{\mat}[1]{{\boldsymbol{\mathbf{#1}}}}

% Hermitian conjugate symbol
\newcommand{\hconj}{\ensuremath{\dagger}}

% Hat scalars
\newcommand{\phat}{\hat{p}}

% Lower case vectors
\newcommand{\va}{\vec{a}}
\newcommand{\vc}{\vec{c}}
\newcommand{\vd}{\vec{d}}
\newcommand{\vk}{\vec{k}}
\newcommand{\vn}{\vec{n}}
\newcommand{\vp}{\vec{p}}
\newcommand{\vr}{\vec{r}}
\newcommand{\vs}{\vec{s}}
\newcommand{\vt}{\vec{t}}
\newcommand{\vu}{\vec{u}}
\newcommand{\vv}{\vec{v}}
\newcommand{\vw}{\vec{w}}
\newcommand{\vx}{\vec{x}}
\newcommand{\vq}{\vec{q}}
\newcommand{\vy}{\vec{y}}
\newcommand{\vb}{\vec{b}}

% Unit vectors
\newcommand{\vbhat}{\hat{\vec{b}}}
\newcommand{\vdhat}{\hat{\vec{d}}}
\newcommand{\vehat}{\hat{\vec{e}}}
\newcommand{\vnhat}{\hat{\vec{n}}}
\newcommand{\vphat}{\hat{\vec{p}}}
\newcommand{\vrhat}{\hat{\vec{r}}}
\newcommand{\vuhat}{\hat{\vec{u}}}
\newcommand{\vvhat}{\hat{\vec{v}}}
\newcommand{\vwhat}{\hat{\vec{w}}}
\newcommand{\vxhat}{\hat{\vec{x}}}
\newcommand{\vyhat}{\hat{\vec{y}}}
\newcommand{\vzhat}{\hat{\vec{z}}}

% Vectors with tildes
\newcommand{\vnt}{\tilde{\vec{n}}}
\newcommand{\vvt}{\tilde{\vec{v}}}
\newcommand{\vxt}{\tilde{\vec{x}}}

% Vectors with bars
\newcommand{\vnb}{\bar{\vec{n}}}
\newcommand{\vvb}{\bar{\vec{v}}}

% Upper case vectors
\newcommand{\vA}{\vec{A}}
\newcommand{\vE}{\vec{E}}
\newcommand{\vH}{\vec{H}}
\newcommand{\vS}{\vec{S}}

% Greek vectors
\newcommand{\veps}{\vec{\varepsilon}}
\newcommand{\vmu}{\vec{\mu}}
\newcommand{\vtheta}{\vec{\theta}}


% Upper case matrices
\newcommand{\mA}{\mat{A}}
\newcommand{\mB}{\mat{B}}
\newcommand{\mC}{\mat{C}}
\newcommand{\mE}{\mat{E}}
\newcommand{\mF}{\mat{F}}
\newcommand{\mG}{\mat{G}}
\newcommand{\mI}{\mat{I}}
\newcommand{\mJ}{\mat{J}}
\newcommand{\mK}{\mat{K}}
\newcommand{\mM}{\mat{M}}
\newcommand{\mN}{\mat{N}}
\newcommand{\mP}{\mat{P}}
\newcommand{\mQ}{\mat{Q}}
\newcommand{\mR}{\mat{R}}
\newcommand{\mS}{\mat{S}}
\newcommand{\mT}{\mat{T}}
\newcommand{\mU}{\mat{U}}
\newcommand{\mV}{\mat{V}}
\newcommand{\mW}{\mat{W}}
\newcommand{\mX}{\mat{X}}
\newcommand{\mY}{\mat{Y}}

\newcommand{\mNh}{\mat{N}^{\scriptscriptstyle -1/2}}

% Matrices with tildes
\newcommand{\mBt}{\tilde{\mat{B}}}
\newcommand{\mCt}{\tilde{\mat{C}}}
\newcommand{\mNt}{\tilde{\mat{N}}}
\newcommand{\mQt}{\tilde{\mat{Q}}}
\newcommand{\mSt}{\tilde{\mat{S}}}

% Matrices denoted by greek letters
\newcommand{\mGamma}{\mat{\Gamma}}
\newcommand{\mLambda}{\mat{\Lambda}}
\newcommand{\mLambdat}{\tilde{\mat{\Lambda}}}  % with a tilde
\newcommand{\mSigma}{\mat{\Sigma}}

%Matrices with bars
\newcommand{\mBb}{\bar{\mat{B}}}
\newcommand{\mCb}{\bar{\mat{C}}}
\newcommand{\mFb}{\bar{\mat{F}}}
\newcommand{\mNb}{\bar{\mat{N}}}
\newcommand{\mSb}{\bar{\mat{S}}}

% Matrices with hats
\newcommand{\mChat}{\hat{\mat{C}}}



%%% Add some new operators
\DeclareMathOperator{\sinc}{sinc}
\DeclareMathOperator{\tri}{tri}
\DeclareMathOperator{\tr}{tr}
\DeclareMathOperator{\Tr}{Tr}
\DeclareMathOperator{\Var}{Var}
\DeclareMathOperator{\Cov}{Cov}
\DeclareMathOperator{\mvec}{vec}
\DeclareMathOperator{\argmax}{argmax}


\newcommand{\brsc}[1]{{\ensuremath{\scriptscriptstyle (#1)}}}


%%% Miscellaneous symbols

\newcommand{\appropto}{\mathrel{\vcenter{
  \offinterlineskip\halign{\hfil$##$\cr
    \propto\cr\noalign{\kern2pt}\sim\cr\noalign{\kern-2pt}}}}}

% Caligraphic letters
\newcommand{\calF}{\mathcal{F}}
\newcommand{\calG}{\mathcal{G}}
\newcommand{\calH}{\mathcal{H}}
\newcommand{\calL}{\mathcal{L}}
\newcommand{\calM}{\mathcal{M}}
\newcommand{\calP}{\mathcal{P}}
\newcommand{\calW}{\mathcal{W}}
\newcommand{\calZ}{\mathcal{Z}}

% Text for denoting max and min
\newcommand{\tmax}{\ensuremath{\text{max}}}
\newcommand{\tmin}{\ensuremath{\text{min}}}

% Binary operators
\newcommand{\kprod}{\otimes}

% Measure for integrating over the sphere
\newcommand{\dhn}{d^2\hat{n}}


%%% Units for cosmology (generally try to use the siunitx packages)
\newcommand{\hMpc}{\;h^{-1}\:\mathrm{Mpc}}
\newcommand{\ihMpc}{\;h\:\mathrm{Mpc}^{-1}}

\newcommand{\eps}{\varepsilon}

\newcommand{\prob}[1]{\mathrm{Pr}(#1)}

% Command to add number to unnumbered equation
% (useful to assign number to final line of align* environment)
\newcommand\numberthis{\addtocounter{equation}{1}\tag{\theequation}}


\usepackage{siunitx}
\sisetup{separate-uncertainty,per-mode=symbol,binary-units}
\DeclareSIUnit\Msolar{M_\odot}
\DeclareSIUnit\degr{deg}
\DeclareSIUnit\parsec{pc}
\DeclareSIUnit\dBm{dBm}
\DeclareSIUnit\jansky{Jy}
\DeclareSIUnit\beam{beam}
\DeclareSIUnit\h{h}
\DeclareSIUnit\GeV{GeV}

\author{Tristan Pinsonneault-Marotte, Richard Shaw}
\title{PHYS 571: Homework 4}
\date{\today}

\renewcommand\diff{\mathrm{d}}

\begin{document}

\maketitle


\begin{itemize}
    \item \textbf{Deadline}: Thursday 21st April, 11:59 PM Pacific Time
    \item All parts carry equal weight.
    \item You \emph{must} work in groups of 2 (if anyone can't find a partner let me know ASAP, and I'll put you into a 3 if needed)
    \item However, you \emph{must} submit your \emph{own} notebook of the assignment and understand all parts of it.
    \item We will look at both submissions for you group, but will only grade one fully, and you will both receive the grade for that one.
\end{itemize}

\section{Zel'dovich Approximation}

In fluid mechanics, there are two common ways to keep track of the properties
and dynamics of the fluid. \emph{Eulerian coordinates} define a fixed grid of
coordinates $\mathbf{x}$ and measure properties of the fluid (e.g. density,
flow, temperature) at those fixed positions. \emph{Lagrangian coordinates}
instead consider coordinates that follow ``fluid elements'' -- small locally
uniform volumes of fluid -- as they move through space with the fluid flow. The
Lagrangian coordinates $\mathbf{q}$, which label fluid elements, therefore
follow a trajectory in time relative to the Eulerian coordinates. This
trajectory is described by a displacement field $\mathbf{\Psi}(\mathbf{q}, t)$:
\begin{equation}
    \mathbf{x} = \mathbf{q} + \mathbf{\Psi}(\mathbf{q}, t) \text{.}
\end{equation}
We can think of this as the path followed by the fluid element that started at
position $\mathbf{q}$, and define $\mathbf{\Psi}(\mathbf{q}, t_i) = 0$.

In the context of cosmological structure formation, we can take the Lagrangian
approach by setting up an array of fluid elements (or ``particles'') along with
initial conditions and following them as they evolve according to the dynamics
of the universe. This is basically how $N$-body cosmological simulations
proceed. In this problem, you will derive the Lagrangian displacement field up
to first order in perturbation theory and use it to make maps of large-scale
structure as it forms in a simulated volume of the universe.

\paragraph{a)} The \emph{physical} velocity field $\mathbf{v} = a\: \diff\mathbf{\Psi}/\diff t$. Use
the equations for the evolution of first order perturbations for pressure-less
matter,
\begin{equation}
\begin{aligned}
    &\frac{\partial \mathbf{v}}{\partial t} + H \mathbf{v} = - \frac{1}{a} \nabla \phi \\
    &\frac{\partial \delta}{\partial t} + \frac{1}{a} \nabla \cdot \mathbf{v} = 0\\
    &\nabla^2 \phi = 4 \pi G \bar{\rho} a^2 \delta \text{,}
\end{aligned}
\label{eq:first-order}
\end{equation}
to integrate in time and derive the following expression for the displacement
field in terms of the gravitational potential and density at an early time
$\phi_0, \bar{\rho}_0$,
\begin{equation}
    \mathbf{\Psi}(\mathbf{q}, t) = \frac{-\nabla \phi_0}{4 \pi G \bar{\rho}_0 a_0^2} D(t)
    \text{,}
\label{eq:za}
\end{equation}
where $a_0 = 1 / (1 + z_0) \neq 1$ and $D(t)$ is the growth factor that describes the evolution of first-order
Eulerian perturbations, $\delta(\mathbf{x}, t) = D(t) \delta_0(\mathbf{x})$. As
$D(t) \propto a(t)$ in a matter-dominated universe, you can assume that $D(t_0)
\ll D(t)$ and use the result
\begin{equation}
    \frac{a^2}{4 \pi \bar{\rho}_0 a_0^3} \frac{\diff D}{\diff t} = \int \diff t \frac{D}{a} \text{.}
\end{equation}
The Zel'dovich approximation consists in using Equation~\ref{eq:za} to calculate
the growth of structure in the universe. It turns out that the density field
obtained from this approach is more robust to non-linear evolution than the
first-order solution that was used to derive it!

\paragraph{b)} \emph{(An accompanying jupyter notebook will be provided with the
code to complete this part. But feel free to try it out on your own!)} Use
CAMB\footnote{\url{camb.readthedocs.io}} to generate the matter power spectrum
at early times ($z=1000$ is a good choice) with the following parameters: $H_0 =
67.0$ km/s/Mpc, $\Omega_b = 0.05$ and $\Omega_c = 0.25$ (you can leave the
others as default). Draw a realisation of 3-dimensional Gaussian density
fluctuations on a 512\times512\times16 grid of $(2\:\mathrm{Mpc})^3$ cells.

\paragraph{c)} From the density contrast realisation you obtained in b), use the
Poisson equation (Equation~\ref{eq:first-order}) to compute the initial
gravitational potential $\phi_0$ and its gradient. Then compute the
displacement field at a few redshift slices ($z \in \{100, 10, 5, 1\}$ or more).
At each of these redshifts, make a plot of the particle positions to see
large-scale structure take form.

\emph{(Bonus)} For a bit of fun, try and colour the particles by whether they came from underdense or overdense regions in the original field.

\emph{(Hint)} There are a few ways you could visualise this (don't hesitate to
be creative), but one is as follows: Start with a set of particle positions
arranged uniformly on a plane in your 3-d space. At each redshift slice,
evaluate the displacement field you computed at the starting position
(Lagrangian coordinate) of each particle and add it to the starting position to
obtain its current position. Make a scatter plot of the positions of the
displaced particles projected onto the initial plane. At low redshifts, you will
see the particles clump together and form the cosmic web! You can play around
with the number of particles in order to get a satisfying visualisation.

\emph{(Hint)} You will need to know how to take the inverse-Laplacian of a field. You can do that by applying $-1 / k^2$ to the field in Fourier space using quantities you already calculated in part b).

% using
% the discrete Fourier transform. Given $N$ discrete samples of a field $\phi_j =
% \phi(x_j), j \in \mathbb{Z}$, their discreet Fourier transform is
% \begin{equation}
%     \tilde{\phi}_n = \sum_j^N e^{-2 \pi i n j / N} \phi_j \text{,}
% \end{equation}
% and the inverse transform is
% \begin{equation}
%     \phi_j = \frac{1}{N}\sum_n^N e^{2 \pi i n j / N} \tilde{\phi}_n \text{.}
% \end{equation}
% In the native integer units, the Laplacian of $\phi$ is
% \begin{equation}
%     \nabla^2 \phi_j = \partial_j^2 \phi_j = \frac{1}{N}\sum_n^N e^{2 \pi i n j / N}
%     \left[- \left(2 \pi \frac{n}{N}\right)^2 \tilde{\phi}_n \right] \text{,}
% \end{equation}
% and it's transform is therefore
% \begin{equation}
%     \tilde{\nabla^2 \phi}_n = - \left(2 \pi \frac{n}{N}\right)^2 \tilde{\phi}_n \text{.}
% \end{equation}

\paragraph{d)} Compute the power spectrum of density fluctuations using the
Zel'dovich approximation at redshift $z=10$ (but feel free to do so at other
redshifts too). On the same plot, show the linear power spectrum (from CAMB).
How do they differ?

\emph{(Hint)} to make this easier we have included two helpful routines that we have tested work. One will take an array of particle positions and assign all their mass into the nearest locations in a regular cartesian grid (note this requires you to install the \texttt{pynbody}\footnote{\url{https://pynbody.github.io/pynbody/index.html}} package. The second takes a regular grid of densities and estimates the 3D power spectrum of that grid. If you decide to use these, you should try and understand how they work.

\emph{Dealing with the finite box size effects with these routines is tricky and I have found that although these routines work accurately on the initial realisation $\delta_0$ on the Zel'dovich realisations they estimate $\sim 3 \times$ more power than the linear power spectrum predicts at all redshifts.}

\paragraph{e)} Outside the local Universe we are unable to directly measure the distance to other galaxies and so we assume that the observed redshift can be directly mapped to the comoving distance via the usual integral relationship.

However, the observed redshift of a galaxy has a contributions from its peculiar motion (c.f. homework 1, question 5) not just the expansion of the Universe, and this biases the inferred radial position. Show that for a galaxy located at \emph{true} comoving position $\vx = \chi \vnhat$, the inferred radial distance to the object $\chi_s$ is
\begin{equation}
    \chi_s = \chi + \frac{1 + z}{H(z)}\vnhat \cdot \vv(\vx)
\end{equation}
where $\vv(\vx)$ is the physical velocity field and $z$ is the ``true'' redshift of the galaxy, and $\vnhat$ is the unit vector along the radial direction.

The inferred position of galaxies defines a new coordinate system $\vs = \chi_s \vnhat$ which is termed \emph{redshift-space}, as opposed to the coordinate system $\vx$ which is real-space. As the peculiar velocities are correlated with the underlying density field, this produce coherent artifacts in the observed distribution of galaxies, known as \emph{redshift-space distortions} or RSDs that must be carefully accounted for in any large-scale structure analysis.

Show that using the Zel'dovich approximation, the position of a particle in \emph{redshift-space} that is at Lagrangian position $\vq$ is
\begin{equation}
\vs = \vq + \left(\mathbf{I} + f \, \vnhat \vnhat^T\right) \mathbf{\Psi}(\vq, t)
\end{equation}
where $f$ is the growth rate
\begin{equation}
f = \frac{d \log D}{d \log a} \; .
\end{equation}
Apply this to your Zel'dovich simulation and compare the apparent structure in redshift-space to that in real-space. Note that the galaxies are located at cosmological distances from the observer, so the line of sight direction vector $\vnhat$ can be approximated as constant and identified with one of the long axes of your simulated volume.

\paragraph{f)} The amount of mass or numbers of galaxies is the same within equivalent volumes in Eulerian, Lagrangian and redshift-space, i.e.
\begin{equation}
    \bar{\rho} d^3q = \bar{\rho} \ls 1 + \delta(\vx)\rs d^3x = \bar{\rho} \ls 1 + \delta_s(\vs)\rs d^3s
\end{equation}
Without assuming any form for $\mathbf{\Psi}$ show that the Eulerian density perturbation is
\begin{equation}
    \delta(\vx) = - \nabla_q \cdot \mathbf{\Psi}(\vq)
\end{equation}
at first order in $\mathbf{\Psi}$, where you might want to use the result that $\det{(\mI + \epsilon \mX)} = 1 + \epsilon \tr{\mX} + O(\epsilon^2)$, and then show that is consistent with the form of $\mathbf{\Psi}$ derived in part a).

Using the results in part e) show that the conservation of mass in equivalent volumes means the density perturbation in redshift-space at first order is given by
\begin{equation}
    \delta_s(\vs) = -\nabla_q \cdot \mathbf{\Psi}(\vq) - f \, \vnhat \cdot \nabla_q (\vnhat \cdot \mathbf{\Psi}(\vq))
\end{equation}
and that when combined with the results of part a) in k-space this is equivalent to
\begin{equation}
    \delta_s(\vk) = \delta(\vk) \lp 1 + f \mu^2 \rp
\end{equation}
where $\mu$ is the cosine of the angle of the Fourier k-vector to the line of site direction, i.e. $\mu = \vnhat \cdot \vk / k$. This is the famous \emph{Kaiser}\footnote{\url{https://ui.adsabs.harvard.edu/abs/1987MNRAS.227....1K/abstract}} result explaining the apparent anisotropy of the observed matter power spectrum.

By a method of your choice, show that your Zel'dovich simulations in redshift-space are statistically anisotropic, but those in real space are not, and discuss how this is consistent with the cosmological principle.


\end{document}