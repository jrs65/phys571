\documentclass[12pt]{article}
\usepackage{amsmath}
\usepackage{amssymb}
\usepackage[T1]{fontenc} % Use 8-bit encoding that has 256 glyphs
\usepackage{lmodern}
\usepackage{graphicx}
\usepackage{hyperref}

\usepackage{siunitx}
\sisetup{separate-uncertainty,per-mode=symbol,binary-units}
\DeclareSIUnit\Msolar{M_\odot}
\DeclareSIUnit\degr{deg}
\DeclareSIUnit\parsec{pc}
\DeclareSIUnit\dBm{dBm}
\DeclareSIUnit\jansky{Jy}
\DeclareSIUnit\beam{beam}
\DeclareSIUnit\h{h}
\DeclareSIUnit\GeV{GeV}

\author{Tristan Pinsonneault-Marotte, Richard Shaw}
\title{PHYS 571: Homework 4}
\date{\today}

\newcommand\diff{\mathrm{d}}

\begin{document}

\maketitle

\begin{itemize}
    \item \textbf{Deadline}: Friday 1st April, 11:59 PM Pacific Time
    \item In this assignment you can set $c = \hbar = k_B = 1$.
    \item The first three questions have equal weight, the final question has twice the weight of the others.
    \item You are strongly encouraged to works in groups of 2 for questions 3 and 4, and you can submit identical notebooks for your solutions (you should each submit a copy though).
\end{itemize}

\section{Zel'dovich Approximation}

In fluid mechanics, there are two common ways to keep track of the properties
and dynamics of the fluid. \emph{Eulerian coordinates} define a fixed grid of
coordinates $\mathbf{x}$ and measure properties of the fluid (e.g. density,
flow, temperature) at those fixed positions. \emph{Lagrangian coordinates}
instead consider coordinates that follow ``fluid elements'' -- small locally
uniform volumes of fluid -- as they move through space with the fluid flow. The
Lagrangian coordinates $\mathbf{q}$, which label fluid elements, therefore
follow a trajectory in time relative to the Eulerian coordinates. This
trajectory is described by a displacement field $\mathbf{\Psi}(\mathbf{q}, t)$:
\begin{equation}
    \mathbf{x} = \mathbf{q} + \mathbf{\Psi}(\mathbf{q}, t) \text{.}
\end{equation}
We can think of this as the path followed by the fluid element that started at
position $\mathbf{q}$, and define $\mathbf{\Psi}(\mathbf{q}, t_i) = 0$.

In the context of cosmological structure formation, we can take the Lagrangian
approach by setting up an array of fluid elements (or ``particles'') along with
initial conditions and following them as they evolve according to the dynamics
of the universe. This is basically how $N$-body cosmological simulations
proceed. In this problem, you will derive the Lagrangian displacement field up
to first order in perturbation theory and use it to make maps of large-scale
structure as it forms in a simulated volume of the universe.

\paragraph{a)} The \emph{physical} velocity field $\mathbf{v} = a\: \diff\mathbf{\Psi}/\diff t$. Use
the equations for the evolution of first order perturbations for pressure-less
matter,
\begin{equation}
\begin{aligned}
    &\frac{\partial \mathbf{v}}{\partial t} + H \mathbf{v} = - \frac{1}{a} \nabla \phi \\
    &\frac{\partial \delta}{\partial t} + \frac{1}{a} \nabla \cdot \mathbf{v} = 0\\
    &\nabla^2 \phi = 4 \pi G \bar{\rho} a^2 \delta \text{,}
\end{aligned}
\label{eq:first-order}
\end{equation}
to integrate in time and derive the following expression for the displacement
field in terms of the gravitational potential and density at an early time
$\phi_0, \bar{\rho}_0$,
\begin{equation}
    \mathbf{\Psi}(\mathbf{q}, t) = \frac{-\nabla \phi_0}{4 \pi G \bar{\rho}_0} D(t)
    \text{,}
\label{eq:za}
\end{equation}
where $D(t)$ is the growth factor that describes the evolution of first-order
Eulerian perturbations, $\delta(\mathbf{x}, t) = D(t) \delta_0(\mathbf{x})$, and
$D(t) \propto a(t)$ in a matter-dominated universe. You can assume that $D(t_0)
\ll D(t)$ and use the result
\begin{equation}
    \frac{a^2}{4 \pi \bar{\rho}_0} \frac{\diff D}{\diff t} = \int \diff t \frac{D}{a} \text{.}
\end{equation}
The Zel'dovich approximation consists in using Equation~\ref{eq:za} to calculate
the growth of structure in the universe. It turns out that the density field
obtained from this approach is more robust to non-linear evolution than the
first-order solution that was used to derive it!

\paragraph{b)} \emph{(An accompanying jupyter notebook will be provided with the
code to complete this part. But feel free to try it out on your own!)} Use
CAMB\footnote{\url{camb.readthedocs.io}} to generate the matter power spectrum
at early times ($z=1000$ is a good choice) with the following parameters: $H_0 =
67.0$ km/s/Mpc, $\Omega_b = 0.05$ and $\Omega_c = 0.25$ (you can leave the
others as default). Draw a realisation of 3-dimensional Gaussian density
fluctuations $\delta_0(k_x, k_y)$ from this power spectrum up to $k < 4$
$h$/Mpc, and Fourier transform it to obtain an initial density contrast field
$\delta_0(x, y)$.

\paragraph{c)} From the density contrast realisation you obtained in b), use the
Poisson equation (Equation~\ref{eq:first-order}) to compute the initial
gravitational potential $\phi_0$ and its gradient. Then compute the
displacement field at a few redshift slices ($z \in \{100, 10, 5, 1\}$ or more).
At each of these redshifts, make a plot of the particle positions to see
large-scale structure take form.

\emph{(Hint)} There are a few ways you could visualise this (don't hesitate to
be creative), but one is as follows: Start with a set of particle positions
arranged uniformly on a plane in your 3-d space. At each redshift slice,
evaluate the displacement field you computed at the starting position
(Lagrangian coordinate) of each particle and add it to the starting position to
obtain its current position. Make a scatter plot of the positions of the
displaced particles projected onto the initial plane. At low redshifts, you will
see the particles clump together and form the cosmic web! You can play around
with the number of particles in order to get a satisfying visualisation.

\emph{(Hint)} You will need to know how to take the Laplacian of a field using
the discrete Fourier transform. Given $N$ discrete samples of a field $\phi_j =
\phi(x_j), j \in \mathbb{Z}$, their discreet Fourier transform is
\begin{equation}
    \tilde{\phi}_n = \sum_j^N e^{-2 \pi i n j / N} \phi_j \text{,}
\end{equation}
and the inverse transform is
\begin{equation}
    \phi_j = \frac{1}{N}\sum_n^N e^{2 \pi i n j / N} \tilde{\phi}_n \text{.}
\end{equation}
In the native integer units, the Laplacian of $\phi$ is
\begin{equation}
    \nabla^2 \phi_j = \partial_j^2 \phi_j = \frac{1}{N}\sum_n^N e^{2 \pi i n j / N}
    \left[- \left(2 \pi \frac{n}{N}\right)^2 \tilde{\phi}_n \right] \text{,}
\end{equation}
and it's transform is therefore
\begin{equation}
    \tilde{\nabla^2 \phi}_n = - \left(2 \pi \frac{n}{N}\right)^2 \tilde{\phi}_n \text{.}
\end{equation}

\paragraph{d)} Compute the power spectrum of density fluctuations using the
Zel'dovich approximation at redshift $z=10$ (but feel free to do so at other
redshifts too). On the same plot, show the linear power spectrum (from CAMB).
How do they differ?

\emph{(Hint)} Do to this you will need to convert your displaced particles back
into a density fluctuation field. You can use the
\texttt{pynbody}\footnote{\url{https://pynbody.github.io/pynbody/index.html}}
package for this. Create a particle catalog using \texttt{pynbody.new} and
populate it with your particle positions. The function
\texttt{pynbody.sph.to\_3d\_grid} can then be used to transform these positions
into a density field $\rho$ on a regular grid. From there you can calculate the
density contrast $\delta$ and Fourier transform it to evaluate the power
spectrum. Pay attention to the normalisation of the latter. You may find the
function \texttt{numpy.bincount} useful to compute the average of the
$\left<\delta^2(\mathbf{k})\right>$ along circles of constant
$\left|\mathbf{k}\right|$.

\end{document}