\documentclass[12pt]{article}
\usepackage{amsmath}
\usepackage{amssymb}
\usepackage[T1]{fontenc} % Use 8-bit encoding that has 256 glyphs
\usepackage{lmodern}
\usepackage{graphicx}

\author{Tristan Pinsonneault-Marotte, Richard Shaw}
\title{PHYS 571: Homework 1}
\date{\today}

\newcommand\diff{\mathrm{d}}

\begin{document}

\maketitle

Deadline: Wednesday 9th February, 11:59 PM Pacific Time

\section{Particle trapped on a sphere}
(Dodelson 2.7, first edition)

Find and apply the metric, Christoffel symbols, and Ricci scalar for a particle
trapped on the surface of a sphere with radius $r$.

\paragraph{a)} Using coordinates $t, \theta, \phi$, the metric is
\begin{equation}
    g_{\mu\nu} = \begin{pmatrix}
        -1 & 0 & 0 \\
        0 & r^2 & 0 \\
        0 & 0 & r^2 \sin\theta
    \end{pmatrix}
    \text{ .}
\end{equation}
Show that the only nonvanishing Christoffel symbols are
$\Gamma^\theta_{\phi\phi}$, $\Gamma^\phi_{\phi\theta}$ and
$\Gamma^\phi_{\theta\phi}$. Express these in terms of $\theta$.

\paragraph{b)} Use these and the geodesic equation to find the equations of
motion for the particle.

\paragraph{c)} Find the Ricci tensor\footnote{You might not be able to do this until next week}. Show that contraction of this tensor leads
to
\begin{equation}
    \mathcal{R} = g^{\mu\nu} R_{\mu\nu} = \frac{2}{r^2} \text{ .}
\end{equation}

\section{Schwarzschild metric}

A spherically symmetric spacetime is realised by the Schwarzschild metric:
\begin{equation}
    \diff s^2 = - \left(1 - \frac{2 G M}{r c^2}\right) c^2 \diff t^2
    + \left(1 - \frac{2 G M}{r c^2}\right)^{-1} \diff r^2
    + r^2 \diff \Omega^2 \text{ ,}
\label{eq:schwarz}
\end{equation}
where $M$ is a constant. This metric describes the empty space on the outside of
a compact spherical distribution of mass at the origin, like a black hole or the
Sun. In that case, $M$ is the total mass of the central object.

\paragraph{a)} Show that for typical distances in the solar system, $\frac{2 G
M}{r c^2} \ll 1$.

In this regime the Schwarzschild metric can be approximated to leading order as
\footnote{It may not be immediately obvious, but a suitable change of coordinates leads to this form.}
\begin{equation}
    \diff s^2 = - \left(1 + \frac{2 \Phi}{c^2}\right) c^2 \diff t^2
    + \left(1 - \frac{2 \Phi}{c^2}\right)
    (\diff x^2 + \diff y^2 + \diff z^2) \text{ ,}
\label{eq:schwarz_pert}
\end{equation}
where we've identified the Newtonian potential $\Phi = -GM/r$.

Find the equation of motion for a massive particle traveling in this field
(Dodelson 2.3). You can use the fact that the particle is non-relativistic, so
$P^0 \gg P^i$.

\paragraph{b)} Show that $\Gamma^0_{00} = c^{-3} \partial \Phi / \partial t$ and
$\Gamma^i_{00} = c^{-2} \delta^{ij} \partial \Phi / \partial x^j$.

\paragraph{c)} Show that the time component of the geodesic equation implies
that energy $P^0 + m \Phi / c$ is conserved.

\paragraph{d)} Show that the space components of the geodesic equation lead to
$\diff P^i / \diff t = - m \delta^{ij} \partial \Phi / \partial x^j$ in
agreement with Newtonian theory.

\section{Gravitational redshift}

The perturbed metric in Equation~\ref{eq:schwarz_pert} makes it clear that time
intervals between events will be dilated according to the value of the
gravitational potential $\Phi$ at the location where they take place. The ratio
of elapsed time at two locations $A$ and $B$ is
\begin{equation}
    \frac{\Delta \tau_A}{\Delta \tau_B} \approx 1 + \frac{\Phi_A - \Phi_B}{c^2} \text.
\end{equation}

This effect was first experimentally demonstrated by Pound and Rebka in 1959. In
a few sentences, describe how the experiment was performed.

\section{Deflection of light}

Consider a photon with a trajectory that grazes the sun, with closest approach
at $y = b$. If it is unperturbed, the photon travels in a straight line,
\begin{equation}
    x = ct, \quad y = b, \quad z = 0 \text{ .}
\end{equation}
In Newtonian gravity, the Sun will deflect the photon from its straight line
path according to the equation of motion
\begin{equation}
    \frac{\diff\mathbf{v}}{\diff t} = \nabla\left(\frac{GM}{r}\right)
    = \left(\frac{GM}{r^2}\right) \left(- \hat{r}\right) \text{.}
\end{equation}
This will perturb the path most significantly in the $y$ direction and we can
assume that the $x$ component of the momentum is essentially unchanged.

\paragraph{a)} In the Newtonian limit, show that the $y$-component of the acceleration is
\begin{equation}
    \frac{\diff v_y}{\diff t} = - G M \frac{y}{\left(y^2 + x^2\right)^{3/2}}
    \approx - G M \frac{b}{\left(b^2 + x^2\right)^{3/2}} \text{ ,}
\end{equation}
where in the second equality we take $y \approx b$ because the deflection is
very small.
% We can integrate this to find the velocity after the photon
% encounters the Sun:
% \begin{equation}
% \begin{split}
%     v_y(x \rightarrow \infty)
%     &= - \int \diff t \frac{G M b}{\left(b^2 + x^2\right)^{3/2}}\\
%     &= - \int_{-\infty}^\infty \diff x \frac{\diff t}{\diff x}
%         \frac{G M b}{\left(b^2 + x^2\right)^{3/2}}\\
%     &= - \frac{G M b}{c} \int_{-\infty}^\infty
%         \frac{1}{\left(b^2 + x^2\right)^{3/2}}\\
%     &= - \frac{2 G M}{b c} \text{.}
% \end{split}
% \end{equation}
Integrate this along the \emph{unperturbed} path to show that the deflection angle is
\begin{equation}
    \Delta \theta = \frac{v_y}{v_x} = - \frac{2 G M}{b} \; .
\end{equation} \\[10pt]
In this rest of this question, you will calculate the deflection angle in GR using the
perturbed metric given above:

\paragraph{b)} Using the geodesic equation for the metric in
Equation~\ref{eq:schwarz_pert}, show that the equation of motion for a \emph{photon} is to leading order
\begin{equation}
    \frac{\diff P^y}{\diff \lambda}
    = \frac{2 M b}{\left(x^2 + b^2\right)^{3/2}} (P^x)^2 \text{ ,}
\end{equation}
where $\lambda$ is an affine parameter for the geodesic.

\paragraph{c)} Integrate the equation of motion from \textbf{b)} to find the
deflection angle.

\paragraph{d)} Calculate the maximum observable deflection angle for light passing close to the sun, and in a few sentences describe how the Eddington expedition confirmed the general relativistic result for the deflection angle.

\section{Redshift in the FRW metric}

Consider a galaxy in the FRW metric located a large distance from an observer.
The galaxy emits a photon at wavelength $\lambda_e$ in its inertial reference
frame at time $t_e$. When the photon reaches the observer, its measured
wavelength is $\lambda_{r}$. We fix the scale factor to be $a=1$ when the
observer measures the photon.

\paragraph{a)} First, let's assume that the galaxy has no peculiar velocity with
respect to the comoving frame defined by the metric. Recall that photons have
four-momentum
\begin{equation}
    P^\mu = \left(E / c, \mathbf{p}\right) \text{ ,}
\end{equation}
where $E = hc / \lambda$, and an observer with four-velocity $U^\mu$ will
measure their energy as
\begin{equation}
    E_r = g_{\mu \nu} U^\mu P^\nu \text{ .}
\end{equation}
Using the fact that the photon travels along a
geodesic, show that the observed wavelength is $\lambda_{r} = \lambda_e /
a(t_e)$.

This relationship defines the redshift $z$
\begin{equation}
    \frac{\lambda_{r}}{\lambda_e} = \frac{1}{a(t_e)} = 1 + z \text{ .}
\end{equation}

\paragraph{b)} Writing a particle's peculiar velocity as $v^i = d x^i / dt$ show that the time component of its four-velocity is
\begin{equation}
    U^0 = \left(1 - a^2(t) \frac{v^i v_i}{c^2}\right)^{-1/2}
\end{equation}
and derive the remaining components.

\paragraph{c)} Show that the full expression for the redshift seen by a comoving observer of a non-comoving galaxy in direction $\hat{\mathbf{n}}$ is
\begin{equation}
    \lambda_r = \lambda_e \: \frac{1}{a(t_e)} \: \left(1 + a(t_e) \frac{\hat{n}_i v^i}{c} \right) \: \frac{1}{\left(1 - a(t_e)^2 v^i v_i / c^2\right)^{1/2}}
\end{equation}
and interpret each term in the above expression.

\end{document}