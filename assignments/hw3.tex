\documentclass[12pt]{article}
\usepackage{amsmath}
\usepackage{amssymb}
\usepackage[T1]{fontenc} % Use 8-bit encoding that has 256 glyphs
\usepackage{lmodern}
\usepackage{graphicx}
\usepackage{hyperref}

\usepackage{siunitx}
\sisetup{separate-uncertainty,per-mode=symbol,binary-units}
\DeclareSIUnit\Msolar{M_\odot}
\DeclareSIUnit\degr{deg}
\DeclareSIUnit\parsec{pc}
\DeclareSIUnit\dBm{dBm}
\DeclareSIUnit\jansky{Jy}
\DeclareSIUnit\beam{beam}
\DeclareSIUnit\h{h}

\author{Tristan Pinsonneault-Marotte, Richard Shaw}
\title{PHYS 571: Homework 3}
\date{\today}

\newcommand\diff{\mathrm{d}}

\begin{document}

\maketitle

\begin{itemize}
    \item \textbf{Deadline}: Wednesday 2nd March, 11:59 PM Pacific Time
    \item In this assignment you can set $c = \hbar = k_B = 1$.
    \item The first three questions have equal weight to the data analysis mini-project.
\end{itemize}



\section{Equilibrium Densities}

\paragraph{a)} Show that in the relativistic limit, $m \ll T$, the number
density of particles is
\begin{equation}
    n = \frac{\zeta(3)}{\pi^2} g T^3
    \begin{cases}
        1 \quad \text{for bosons} \\
        \frac{3}{4} \quad \text{for fermions}
    \end{cases}
\end{equation}
and the energy density is
\begin{equation}
    \rho = \frac{\pi^2}{30} g T^4
    \begin{cases}
        1 \quad \text{for bosons} \\
        \frac{7}{8} \quad \text{for fermions}
    \end{cases}
    \text{.}
\end{equation}
You can assume that the chemical potential $\mu \ll T$.

You will need the following standard integrals
\begin{equation}
    \int_0^{\infty} \diff x~\frac{x^n}{e^x - 1} = \zeta(n + 1) \Gamma(n + 1) \text{,}
\end{equation}
\begin{equation}
    \int_0^{\infty} \diff x~x^n e^{-x^2} =
    \frac{1}{2} \Gamma\left( \frac{1}{2} (n + 1) \right) \text{,}
\end{equation}
where $\zeta(x)$ is the Riemann zeta function.

\emph{Hint:} Note that
\begin{equation}
    \frac{1}{e^x + 1} = \frac{1}{e^x - 1} - \frac{2}{e^{2x} - 1} \text{.}
\end{equation}

\paragraph{b)} Show that in the non-relativistic limit, $m \gg T$, the number
density of particles is
\begin{equation}
    n = g \left(\frac{m T}{2 \pi}\right)^{3/2} e^{(\mu-m)/T} \text{,}
\label{eq:nr-density}
\end{equation}
where you can use that $m - \mu \gg T$.

Show that the energy density is to leading order in $p^2 / (2m)$
\begin{equation}
    \rho = m n + \frac{3}{2} n T \text{.}
\end{equation}


\section{Chemical Potential}

\paragraph{a)} Given the temperature of the CMB today, $T_0 = 2.7$ K, calculate
the baryon-to-photon ratio
\begin{equation}
    \eta = \frac{n_b}{n_\gamma}
\end{equation}
in terms of the baryon density today, $\Omega_b h^2$. Evaluate your result for
$\Omega_b=0.05,\ h=0.7$.

\paragraph{b)} (Baumann Problem Set 2.3)
Show that the difference between the number densities of electrons and positrons
in the relativistic limit ($m_e \ll T$) is
\begin{equation}
    n_e - \bar{n}_e \approx \frac{gT^3}{6 \pi^2}\left[\pi^2
    \left(\frac{\mu_e}{T}\right) + \left(\frac{\mu_e}{T}\right)^3
    \right] \text{,}
\end{equation}
where $\mu_e$ is the chemical potential.

The electrical neutrality of the universe implies that the number of protons
$n_p = n_e - \bar{n}_e$. Use this and your result for the brayon-to-photon ratio
to estimate $\mu_e/T$.

\paragraph{c)} Now consider the chemical potential in the non-relativistic limit
($m \gg T$). To derive Equation~\ref{eq:nr-density} we had assumed that $m - \mu
\gg T$ in this regime. Convert this limit to a constraint on the number density.
Interpret this constraint by restoring the physical constants to the equation.


\section{Saha Equation}
calculate some equlibrium densities for recombination


\section{}

\end{document}