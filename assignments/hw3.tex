\documentclass[12pt]{article}
\usepackage{amsmath}
\usepackage{amssymb}
\usepackage[T1]{fontenc} % Use 8-bit encoding that has 256 glyphs
\usepackage{lmodern}
\usepackage{graphicx}
\usepackage{hyperref}

\usepackage{siunitx}
\sisetup{separate-uncertainty,per-mode=symbol,binary-units}
\DeclareSIUnit\Msolar{M_\odot}
\DeclareSIUnit\degr{deg}
\DeclareSIUnit\parsec{pc}
\DeclareSIUnit\dBm{dBm}
\DeclareSIUnit\jansky{Jy}
\DeclareSIUnit\beam{beam}
\DeclareSIUnit\h{h}
\DeclareSIUnit\GeV{GeV}

\author{Tristan Pinsonneault-Marotte, Richard Shaw}
\title{PHYS 571: Homework 3}
\date{\today}

\newcommand\diff{\mathrm{d}}

\begin{document}

\maketitle

\begin{itemize}
    \item \textbf{Deadline}: Wednesday 2nd March, 11:59 PM Pacific Time
    \item In this assignment you can set $c = \hbar = k_B = 1$.
    \item The first three questions have equal weight to the data analysis mini-project.
\end{itemize}



\section{Equilibrium Densities}

\paragraph{a)} Show that in the relativistic limit, $m \ll T$, the number
density of particles is
\begin{equation}
    n = \frac{\zeta(3)}{\pi^2} g T^3
    \begin{cases}
        1 \quad \text{for bosons} \\
        \frac{3}{4} \quad \text{for fermions}
    \end{cases}
\end{equation}
and the energy density is
\begin{equation}
    \rho = \frac{\pi^2}{30} g T^4
    \begin{cases}
        1 \quad \text{for bosons} \\
        \frac{7}{8} \quad \text{for fermions}
    \end{cases}
    \text{.}
\end{equation}
You can assume that the chemical potential $\mu \ll T$.

You will need the following standard integrals
\begin{equation}
    \int_0^{\infty} \diff x~\frac{x^n}{e^x - 1} = \zeta(n + 1) \Gamma(n + 1) \text{,}
\end{equation}
\begin{equation}
    \int_0^{\infty} \diff x~x^n e^{-x^2} =
    \frac{1}{2} \Gamma\left( \frac{1}{2} (n + 1) \right) \text{,}
\end{equation}
where $\zeta(x)$ is the Riemann zeta function.

\emph{Hint:} Note that
\begin{equation}
    \frac{1}{e^x + 1} = \frac{1}{e^x - 1} - \frac{2}{e^{2x} - 1} \text{.}
\end{equation}

\paragraph{b)} Show that in the non-relativistic limit, $m \gg T$, the number
density of particles is
\begin{equation}
    n = g \left(\frac{m T}{2 \pi}\right)^{3/2} e^{(\mu-m)/T} \text{,}
\label{eq:nr-density}
\end{equation}
where you can use that $m - \mu \gg T$.

Show that the energy density is to leading order in $p^2 / (2m)$
\begin{equation}
    \rho = m n + \frac{3}{2} n T \text{.}
\end{equation}


\section{Chemical Potential}

\paragraph{a)} Given the temperature of the CMB today, $T_0 = 2.7$ K, calculate
the baryon-to-photon ratio
\begin{equation}
    \eta = \frac{n_b}{n_\gamma}
\end{equation}
in terms of the baryon density today, $\Omega_b h^2$. Evaluate your result for
$\Omega_b=0.05,\ h=0.7$.

\paragraph{b)} (Baumann Problem Set 2.3)
Show that the difference between the number densities of electrons and positrons
in the relativistic limit ($m_e \ll T$) is
\begin{equation}
    n_e - \bar{n}_e \approx \frac{gT^3}{6 \pi^2}\left[\pi^2
    \left(\frac{\mu_e}{T}\right) + \left(\frac{\mu_e}{T}\right)^3
    \right] \text{,}
\end{equation}
where $\mu_e$ is the chemical potential.

The electrical neutrality of the universe implies that the number density of
protons $n_p = n_e - \bar{n}_e$. Use this and your result for the
brayon-to-photon ratio to estimate $\mu_e/T$.

\paragraph{c)} Now consider the chemical potential in the non-relativistic limit
($m \gg T$). To derive Equation~\ref{eq:nr-density} we had assumed that $m - \mu
\gg T$ in this regime. Convert this limit to a constraint on the number density.
Interpret this constraint by restoring the physical constants to the equation.


\section{Hydrogen Recombination}

At temperatures $T \gtrsim 1$~eV, photons, electrons, and protons remained
tightly coupled via electromagnetic interactions. Eventually, the temperature
dropped sufficiently for electrons and protons to combine into neutral hydrogen,
a process called recombination. As the fraction of free electrons got smaller,
photon scattering became increasingly rare, until the mean free path for photons
was larger than the horizon distance and the universe became transparent. From
then on, photons could travel unimpeded, and they reach us today as the cosmic
microwave background.

Recombination proceeds through the following reaction:
\begin{equation}
    e^- + p^+ \rightarrow H + \gamma \text{.}
\end{equation}
At $T \sim 1$~eV, the electrons, protons, and hydrogen are all non-relativistic
and in thermal and chemical equilibrium. Considering the equilibrium densities
of all of these components, one arrives at the Saha equation (see Baumann 3.3
for the derivation):
\begin{equation}
    \left(\frac{1 - X_e}{X_e^2}\right)_{eq} = \frac{2 \zeta(3)}{\pi^2}
    \eta \left(\frac{2 \pi T}{m_e}\right)^{3/2} e^{B_H/T} \text{,}
\end{equation}
where $X_e = n_e / n_b$ is the fraction of free electrons to baryons ($n_b = n_p
+ n_H$) and $B_H$ is the binding energy of hydrogen.

In reality, electrons don't remain in equilibrium forever, and the Saha equation
becomes invalid. When the rate of expansion becomes greater than the rate of
recombination, the density of electrons ``freezes out''. The deviation from
equilibrium is described by the Boltzmann equation
\begin{equation}
    a^{-3} \frac{\diff (n_e a^3)}{\diff t} = - \left<\sigma v\right>
    \left(n_e^2 - (n_e^{eq})^2\right) \text{,}
\end{equation}
where $\left<\sigma v\right>$ is the thermally averaged cross-section for
recombination.

\paragraph{a)} Approximating
\begin{equation}
    \left<\sigma v\right> \approx \sigma_T \left(\frac{B_H}{T}\right)^{1/2} \text{,}
\end{equation}
where $\sigma_T$ is the cross-section for Thompson scattering, and using that
$n_b a^3$ is constant, derive the non-equilibrium equation for recombination
\begin{equation}
    \frac{\diff X_e}{\diff x} = -\frac{\lambda}{x^2}
    \left(X_e^2 - (X_e^{eq})^2\right) \text{.}
    \label{eq:recomb}
\end{equation}
In the equation above, we've defined
\begin{equation}
    x \equiv \frac{B_H}{T} \text{,}
\end{equation}
and
\begin{equation}
    \lambda \equiv \left[\frac{n_b\left<\sigma v\right> x}{H}\right]_{x=1}
    = 5.3 \times 10^4 \left(\frac{\Omega_b h}{0.03}\right)
\end{equation}

\emph{Hint: You can use the fact that the universe is matter dominated during recombination, and for the bulk of the expansion thereafter.}

\paragraph{b)} Numerically integrate Equation~\ref{eq:recomb} to produce a plot
of $X_e$ against temperature showing recombination and the freeze-out of
electrons. On the same figure, plot the prediction of the Saha equation.

[\emph{Hint: for this problem, and the next section, your new favourite routine \texttt{scipy.integrate.solve\_ivp} is an excellent way to do this calculation. However, you may need to change the method from the default as the problem appears to be ``stiff''. I have found the \texttt{Radau} method with an \texttt{atol} of $10^{-15}$ to be a stable, accurate and fast configuration.}]


\section{Dark Matter}

\emph{Baumann section 3.3.2 will be a good reference for this section.}
\newcommand{\sigv}{\langle \sigma v \rangle}

WIMPs are heavy particles that interact primarily via the weak interaction in the early Universe. They somewhat generically produce relic densities that make them an ideal explanation for dark matter.

We assume that in the early Universe WIMPs have an annihilation interaction to a light standard model particle
\begin{equation}
    X + \bar{X} \leftrightarrow l + \bar{l} \; .
\end{equation}
This allows them to maintain equilibrium with the rest of the primordial plasma until they freeze out. This reaction has an unknown cross section $\sigv$.

\paragraph{a)} Construct a Boltzmann equation representing this interaction and solve for the comoving abundance $N_x = k_B n_x / s$\footnote{$s a^3 / k_B$ is roughly the comoving number density of other relativistic particles.}, and plot the abundance against temperature for a \SI{10}{\GeV} WIMP with $\sigv = \SI{1e-26}{\centi\metre^3\per\second}$.

As before you may find it easier to setup your solutions in terms of $x = M_X / T$ and with a generic parameter $\lambda$, and then map the desired parameters into this parameterisation.

\paragraph{b)} To describe abundances in the present Universe we can use the fraction of the critical density $\Omega_i = \rho_i / \rho_\text{crit}$. However, when quoting measurements it is often more useful to use $\Omega_i h^2$ (where $h$ is defined by $H_0 = 100 h\: \si{\kilo\metre\per\second\per\mega\parsec}$) as this removes the dependence on the relatively inaccurate measurements of $H_0$.

By numerically solving the Boltzmann equation, make a plot of the scaled fraction of the critical density today for WIMP dark matter $\Omega_{dm} h^2 = M_X n_X^0 h^2 / \rho_\text{crit}^0$ showing how it varies with $\sigv$ from \SIrange{1e-28}{1e-24}{\centi\metre^3 \per\second} with a fixed particle pass of $M_X = \SI{100}{\GeV}$.

Indicate the ranges that are compatible with the observed dark matter abundance from Planck $\Omega_c h^2 = 0.1206 \pm 0.0021$, and those that explain it fully.

[\emph{Bonus: if you're feeling ambitious you could show the abundance when varying both $\sigv$ and $M_X$ (from \SIrange{1}{100}{\GeV}) simultaneously in a heatmap. In this case you'll need to evaluate the freeze out density for a large number of parameter combinations. However, the freeze out number density depends only on $\lambda$. Try tabulating this for a large range of $\lambda$ values and then use an interpolation over this table to get the results for the values of $M_X$ and $\sigv$ you are interested in. There is no credit beyond kudos for this one.}]

\paragraph{c)} Neutrinos may give an alternative model for dark matter. Assuming that all neutrinos decouple at the same time estimate the abundance today in cosmic neutrinos assuming the normal neutrino hierarchy which has two light neutrinos and one heavy one with a mass $m_{\nu_3} \approx \SI{0.05}{\mega\electronvolt}$.

If there was a fourth neutrino flavour which was substantially heavier, how heavy would it need to be to explain the observed dark matter abundance?

\paragraph{d)} After decoupling particles \emph{free stream} in a straight line without any interaction until today. The distance that they can traverse in that time is called the \emph{free streaming length}. Any initial perturbation on scales smaller than this length are wiped out.

There are two periods of interest:
\begin{itemize}
    \item If a particle is relativistic after decoupling at $t_d$ it propagates as a nearly massless particle until $t_{nr}$ when it becomes non-relativistic.
    \item After it becomes non-relativistic it continues to propagate while it's velocity redshifts away. After matter-radiation equality the mapping between time and scale factor changes and so only the comoving distance moved up until this era $a_{eq}$ is relevant.
\end{itemize}
Calculate the comoving distance for each of these contributions. For the first you can assume that the particle as massless all the way until $t_{nr}$, for the second you can assume that the physical velocity is $c$ at $t_{nr}$ and then redshifts away in the usual manner. You will find it easier to directly use the fact that $a \propto t^{1/2}$ during radiation domination and integrating in time (rather than changing your integration variables to $a$).

Combine them to show that
\begin{equation}
    \chi_{fs} = \frac{2 c t_{nr}}{a_{nr}} \left[1 - \left(\frac{t_d}{t_{nr}}\right)^{1/2} + \ln{\left(\frac{a_{eq}}{a_{nr}}\right)}\right]
\end{equation}
and use this result to
\begin{itemize}
    \item Calculate the free streaming length in Mpc for a plausible set of dark matter parameters (or the whole range above).
    \item Calculate the free streaming length for a massive neutrino dark matter model.
\end{itemize}
Compare both answers to the scales of objects in the current Universe.


\end{document}