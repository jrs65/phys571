\documentclass[12pt]{article}
\usepackage{amsmath}
\usepackage{amssymb}
\usepackage[T1]{fontenc} % Use 8-bit encoding that has 256 glyphs
\usepackage{lmodern}
\usepackage{graphicx}
\usepackage{hyperref}
%!TEX root=beamforming.tex

%%% Standard math package imports
\usepackage{amsmath}
\usepackage{bm}
\usepackage{mathtools}



%%% Derivatives
\newcommand\pdiff[1]{\frac{\partial}{\partial #1}}
\newcommand\diff[2]{\frac{d #1}{d #2}}
\newcommand{\curl}{\,\mbox{curl}\,}
\newcommand\del{\nabla}
\newcommand{\grad}[1][]{\ensuremath{\mathbf{\nabla} #1}}



%% Macros for inserting various types of brackets
\newcommand{\la}{\ensuremath{\left\langle}}  % Angle brackets
\newcommand{\ra}{\ensuremath{\right\rangle}}

\newcommand{\lv}{\ensuremath{\left\lvert}}   % Vertical bars
\newcommand{\rv}{\ensuremath{\right\rvert}}

\newcommand{\ls}{\ensuremath{\left[}}        % Square brackets
\newcommand{\rs}{\ensuremath{\right]}}

\newcommand{\lp}{\ensuremath{\left(}}        % Parentheses
\newcommand{\rp}{\ensuremath{\right)}}

%% Macros for equations
\def\beq{\begin{equation}}
\def\eeq{\end{equation}}


%%% Macros for linear algebra notation
\renewcommand{\vec}[1]{{\boldsymbol{\mathbf{#1}}}}
\newcommand{\mat}[1]{{\boldsymbol{\mathbf{#1}}}}

% Hermitian conjugate symbol
\newcommand{\hconj}{\ensuremath{\dagger}}

% Hat scalars
\newcommand{\phat}{\hat{p}}

% Lower case vectors
\newcommand{\va}{\vec{a}}
\newcommand{\vc}{\vec{c}}
\newcommand{\vd}{\vec{d}}
\newcommand{\vk}{\vec{k}}
\newcommand{\vn}{\vec{n}}
\newcommand{\vp}{\vec{p}}
\newcommand{\vr}{\vec{r}}
\newcommand{\vs}{\vec{s}}
\newcommand{\vt}{\vec{t}}
\newcommand{\vu}{\vec{u}}
\newcommand{\vv}{\vec{v}}
\newcommand{\vw}{\vec{w}}
\newcommand{\vx}{\vec{x}}
\newcommand{\vq}{\vec{q}}
\newcommand{\vy}{\vec{y}}
\newcommand{\vb}{\vec{b}}

% Unit vectors
\newcommand{\vbhat}{\hat{\vec{b}}}
\newcommand{\vdhat}{\hat{\vec{d}}}
\newcommand{\vehat}{\hat{\vec{e}}}
\newcommand{\vnhat}{\hat{\vec{n}}}
\newcommand{\vphat}{\hat{\vec{p}}}
\newcommand{\vrhat}{\hat{\vec{r}}}
\newcommand{\vuhat}{\hat{\vec{u}}}
\newcommand{\vvhat}{\hat{\vec{v}}}
\newcommand{\vwhat}{\hat{\vec{w}}}
\newcommand{\vxhat}{\hat{\vec{x}}}
\newcommand{\vyhat}{\hat{\vec{y}}}
\newcommand{\vzhat}{\hat{\vec{z}}}

% Vectors with tildes
\newcommand{\vnt}{\tilde{\vec{n}}}
\newcommand{\vvt}{\tilde{\vec{v}}}
\newcommand{\vxt}{\tilde{\vec{x}}}

% Vectors with bars
\newcommand{\vnb}{\bar{\vec{n}}}
\newcommand{\vvb}{\bar{\vec{v}}}

% Upper case vectors
\newcommand{\vA}{\vec{A}}
\newcommand{\vE}{\vec{E}}
\newcommand{\vH}{\vec{H}}
\newcommand{\vS}{\vec{S}}

% Greek vectors
\newcommand{\veps}{\vec{\varepsilon}}
\newcommand{\vmu}{\vec{\mu}}
\newcommand{\vtheta}{\vec{\theta}}


% Upper case matrices
\newcommand{\mA}{\mat{A}}
\newcommand{\mB}{\mat{B}}
\newcommand{\mC}{\mat{C}}
\newcommand{\mE}{\mat{E}}
\newcommand{\mF}{\mat{F}}
\newcommand{\mG}{\mat{G}}
\newcommand{\mI}{\mat{I}}
\newcommand{\mJ}{\mat{J}}
\newcommand{\mK}{\mat{K}}
\newcommand{\mM}{\mat{M}}
\newcommand{\mN}{\mat{N}}
\newcommand{\mP}{\mat{P}}
\newcommand{\mQ}{\mat{Q}}
\newcommand{\mR}{\mat{R}}
\newcommand{\mS}{\mat{S}}
\newcommand{\mT}{\mat{T}}
\newcommand{\mU}{\mat{U}}
\newcommand{\mV}{\mat{V}}
\newcommand{\mW}{\mat{W}}
\newcommand{\mX}{\mat{X}}
\newcommand{\mY}{\mat{Y}}

\newcommand{\mNh}{\mat{N}^{\scriptscriptstyle -1/2}}

% Matrices with tildes
\newcommand{\mBt}{\tilde{\mat{B}}}
\newcommand{\mCt}{\tilde{\mat{C}}}
\newcommand{\mNt}{\tilde{\mat{N}}}
\newcommand{\mQt}{\tilde{\mat{Q}}}
\newcommand{\mSt}{\tilde{\mat{S}}}

% Matrices denoted by greek letters
\newcommand{\mGamma}{\mat{\Gamma}}
\newcommand{\mLambda}{\mat{\Lambda}}
\newcommand{\mLambdat}{\tilde{\mat{\Lambda}}}  % with a tilde
\newcommand{\mSigma}{\mat{\Sigma}}

%Matrices with bars
\newcommand{\mBb}{\bar{\mat{B}}}
\newcommand{\mCb}{\bar{\mat{C}}}
\newcommand{\mFb}{\bar{\mat{F}}}
\newcommand{\mNb}{\bar{\mat{N}}}
\newcommand{\mSb}{\bar{\mat{S}}}

% Matrices with hats
\newcommand{\mChat}{\hat{\mat{C}}}



%%% Add some new operators
\DeclareMathOperator{\sinc}{sinc}
\DeclareMathOperator{\tri}{tri}
\DeclareMathOperator{\tr}{tr}
\DeclareMathOperator{\Tr}{Tr}
\DeclareMathOperator{\Var}{Var}
\DeclareMathOperator{\Cov}{Cov}
\DeclareMathOperator{\mvec}{vec}
\DeclareMathOperator{\argmax}{argmax}


\newcommand{\brsc}[1]{{\ensuremath{\scriptscriptstyle (#1)}}}


%%% Miscellaneous symbols

\newcommand{\appropto}{\mathrel{\vcenter{
  \offinterlineskip\halign{\hfil$##$\cr
    \propto\cr\noalign{\kern2pt}\sim\cr\noalign{\kern-2pt}}}}}

% Caligraphic letters
\newcommand{\calF}{\mathcal{F}}
\newcommand{\calG}{\mathcal{G}}
\newcommand{\calH}{\mathcal{H}}
\newcommand{\calL}{\mathcal{L}}
\newcommand{\calM}{\mathcal{M}}
\newcommand{\calP}{\mathcal{P}}
\newcommand{\calW}{\mathcal{W}}
\newcommand{\calZ}{\mathcal{Z}}

% Text for denoting max and min
\newcommand{\tmax}{\ensuremath{\text{max}}}
\newcommand{\tmin}{\ensuremath{\text{min}}}

% Binary operators
\newcommand{\kprod}{\otimes}

% Measure for integrating over the sphere
\newcommand{\dhn}{d^2\hat{n}}


%%% Units for cosmology (generally try to use the siunitx packages)
\newcommand{\hMpc}{\;h^{-1}\:\mathrm{Mpc}}
\newcommand{\ihMpc}{\;h\:\mathrm{Mpc}^{-1}}

\newcommand{\eps}{\varepsilon}

\newcommand{\prob}[1]{\mathrm{Pr}(#1)}

% Command to add number to unnumbered equation
% (useful to assign number to final line of align* environment)
\newcommand\numberthis{\addtocounter{equation}{1}\tag{\theequation}}


\usepackage{siunitx}
\sisetup{separate-uncertainty,per-mode=symbol,binary-units}
\DeclareSIUnit\Msolar{M_\odot}
\DeclareSIUnit\degr{deg}
\DeclareSIUnit\parsec{pc}
\DeclareSIUnit\dBm{dBm}
\DeclareSIUnit\jansky{Jy}
\DeclareSIUnit\beam{beam}
\DeclareSIUnit\h{h}
\DeclareSIUnit\GeV{GeV}

\author{Tristan Pinsonneault-Marotte, Richard Shaw}
\title{PHYS 571: Homework 5}
\date{\today}

\renewcommand\diff{\mathrm{d}}

\begin{document}

\maketitle


\begin{itemize}
    \item \textbf{Deadline}: Thursday 5th June, 11:59 PM Pacific Time
    \item All parts carry equal weight.
    \item Please work in the same group you did for homework 4.
    \item However, you \emph{must} submit your \emph{own} notebook of the assignment and understand all parts of it.
    \item We will look at both submissions for you group, but will only grade one fully, and you will both receive the grade for that one.
\end{itemize}

\section{The Cosmic Microwave Background}

The CMB power spectrum primarily is a probe of the apparent angular distribution of matter on the last scattering surface.

\paragraph{a)} There are three key ingredients controlling the CMB power spectrum, the sound horizon at recombination, and the diffusion damping scale, as well as a key time, the epoch of matter radiation equality. Taking the CMB temperature as fixed, explain why fixing these ingredients is equivalent to fixing $\Omega_b h^2$ and $\Omega_c h^2$.

\paragraph{b)} The angular scale in the CMB corresponding to a fixed physical scale at recombination is constant for all cosmological parameters that give the same angular diameter distance to the last scattering surface.

While fixing $\Omega_b h^2$ and $\Omega_c h^2$ to the Planck best fit values, find at least three additional combinations of $\Omega_k$ and $H_0$ that give the the same distance to last scattering as the Planck best fit parameters. These should be between $\Omega_k = -0.3$ and $\Omega_k = 0$. You may want to use the routine you wrote to calculate $S_k(\chi(z))$ in homework 2.

\paragraph{c)} While again keeping $\Omega_b h^2$ and $\Omega_c h^2$ fixed and all other parameters with the Planck best fit values, make a plot of the CMB power spectrum for the different combinations of $H_0$ and $\Omega_k$ you found in part b). Above $\ell \sim 30$ these should look indistinguishable. This is the \emph{geometric degeneracy} of the CMB.

Explain the origin of the differences at low-$\ell$.

\noindent \emph{Hint: to calculate the power spectrum you should use the CAMB package you installed for Homework 4. You can calculate the CMB power spectrum like this:}
\begin{verbatim}
    import camb
    cparams = camb.CAMBparams(
        ombh2=0.05 * 0.67**2, omch2=0.25 * 0.67**2, H0=67.0, omk=0.0
    )
    res = camb.get_results(cparams)
    Cl_all = res.get_cmb_power_spectra(
        lmax=1000, spectra=("total",), CMB_unit="K"
    )
    Cl_temp = Cl_all["total"][:, 0]
\end{verbatim}
\emph{which gets the temperature power spectrum in Kelvin.}

\end{document}